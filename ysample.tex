\documentclass[横,B5,禁則,名前]{genkou}
\begin{document}
\chapter{コマンド・環境}
\section{コマンド}
\subsection{欧文対応}
\一字下げ
\oubun{\TeX}は優れた行分割アルゴリズムで知られています。
ところが原稿用紙は、それを無効化してわざわざ固定ピッチで
処理させようとするものです。
このような目的に\oubun{\TeX}を使用することに反対される向きも
多いことでしょう。

特に\oubun{\TeX}の基本フォントであるコンピュータモダンローマンなどの
欧文フォントは、文字の幅がキャラクタごとに異なるプロポーショナルフォント
なので、原稿用紙の中で使えば欧文が桝目に合わないのみならず、
その後ろの和文も桝目からずれてしまい、非常に見苦しい結果になります。

本マクロでは、せめて後ろの和文が桝目に合うように、
\oubun{$\backslash$oubun}コマンドを用意しました。
このコマンドは、クラスオプションに「禁則」を指定している場合にのみ
有効です。\空行

\subsection{ルビ}
日本語ではルビを使うことがあります。
これまでにもいろいろな\oubun{\TeX}のルビマクロが作成されてきましたが、
原稿用紙の場合、文字とルビを同じ桝目におさめる必要があるため
本マクロでは専用の\oubun{$\backslash$ruby}コマンドを用意しました。
このマクロでは、一度に6文字までルビを振ることができます。
一度に7文字以上続けてルビを振りたい場合は、いったん6文字以内で
切り分けてアステリスク付きの\oubun{$\backslash$ruby*}を使って
続けてください。
段落の最初の文字にルビを振る場合にも、同様に\oubun{$\backslash$ruby*}を
使ってください。

ルビは、文字ごとに半角カンマで区切ってください。\空行

例:\par
\ruby*{難読語}{なん,どく,ご}には、ルビを\ruby{振}{ふ}る。
\空行

\section{環境}
\subsection{連番箇条書き}
数字の連番付きの箇条書き環境として\oubun{\LaTeX}の
標準クラスと同名の\oubun{enumerate}環境を用意しました。

\begin{enumerate}
    \item 数字の連番付き箇条書きです。
    \item ネスティングすると変ですから、やめましょう。
\end{enumerate}
\空行

\subsection{丸の付いた箇条書き}
名前と使い方はこれも\oubun{\LaTeX}の標準クラスと同じです。

\begin{itemize}
    \item 丸の付いた箇条書きです。
    \item これもネスティングは無しね。
\end{itemize}
\空行

\subsection{見出し付き箇条書き}
標準クラスの\oubun{description}環境に似た見出し付きの箇条書きです。
オプションに文字列を指定すると、それがラベルの幅になります。

\begin{kajogaki}[ああああああ]
    \item[見出し1]一行目
    \item[見出し2]2行目
\end{kajogaki}

\end{document}
